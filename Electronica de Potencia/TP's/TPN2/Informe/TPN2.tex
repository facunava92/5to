\documentclass{article}
\usepackage{graphicx}
%permite ecribir acentos directamente
\usepackage[utf8]{inputenc}
% Esto es para que el LÁTEX sepa que el texto está en español, se agrega el ingles para el paquete de gráfico de circuitos:
\usepackage[spanish]{babel}

\usepackage{geometry}
 \geometry{a4paper,total={170mm,257mm},left=15mm,right=15mm,top=20mm,}
 
\begin{document}

\begin{titlepage}
 \centering
	\includegraphics[scale=0.80]{imagenes/LOGO.jpg} \par
 	\vspace{1cm}
 	{\scshape\LARGE Universidad Tecnológica Nacional \par}
 	{\scshape\large Facultad Regional de Córdoba \par}
 	\vspace{1cm}
	{\bfseries \Large Trabajo Práctico De Laboratorio $N^{\circ} 2$\par}
	{\bfseries \Large Control de ángulo de conducción de un SCR \par}
 	\vspace{1.5cm}

	\begin{tabular}{ll}
		Alassia, Francisco	&	60861	\\
		Amaya, Matías		&	68284	\\
		Lamas, Matías 		&	65536 	\\
		Navarro, Facundo 	&	63809 	\\
		Veron, Misael	 	&	62628
	\end{tabular}
	
	\vspace{1cm}
	Curso: 5r2 \\
	Grupo $N^{\circ} 11$
 	\vfill
	{\bfseries \Large Electrónica de Potencia \par}

	\vspace{1.5cm}
	Docentes: \par
	Ing. Oros \par
	Ing. Rabinovich \par

 	\vfill
	{\large \today\par}
\end{titlepage}



\section{Introducción}
Diseñar y construir un circuito para el control del ángulo de conducción de un SCR mediante el método escalón-rampa coseno. La tensión en la carga debe ser controlada por una señal de reverencia $V_{reff}$ , que variará entre 0 y 10 $V$.


\subsection{Funcionamiento}
La figura \textcolor{blue}{\ref{fig:figura1}}corresponde al circuito de control del ángulo de conducción de onda completa de un SCR con disparo por UJT. El circuito del método escalón-rampa coseno de la figura , deriva del circuito anterior con el agregado de algunas variantes que permite mayor linealidad entre la tensión de referencia y la tensión en la carga.

\begin{figure}[h]
 \begin{center}
	\includegraphics[scale=0.6]]{imagenes/circuit_jfet}
	\caption{Circuito en Orcad PSpice}\label{fig:circuit_jfet}
 \end{center}
\end{figure}


\section{Conclusion}
Usando VIM en Windows 10

\end{document}
